Na dodaném substrátu (č. 4) jsme za pomoci Čtyřbodové metody měřili vrstvový odpor různých TLV vzorků. S použitím korekčních koeficientů jsme získali hodnoty v rozsahu přibližně 14 -- \qty{21}{\kilo\ohm\per sq}. Jestli je rozptyl způsoben skutečně různými hodnotami a nebo nedostatečnou korekcí by bylo nutné ověřit měřením jinou metodou popř. více vzorků. Vzorek 2D nebylo možné stanovit, protože zde již neplatí podmínka geometrických rozměrů -- pro použití této metody by měly rozměry vzorků být výrazně větší než rozestup měřících hrotů. 

Dále jsme u dvou dodaných TLV rezistorů měřili TKR, pro široký (a krátký) rezistor jsme došli k \(TKR_{S} \doteq \qty{-2,404e-3}{\percent\per\degreeCelsius}\), jedná se tedy o negativní TKR. Pro dlouhý (a úzký) rezistor jsme naopak stanovili \(TKR_{D} \doteq \qty{1,178e-2}{\percent\per\degreeCelsius}\). Protože mě nenapadá žádné odůvodnění toho, proč by měl tvar rezistoru ovlivnit jeho TKR, nabízí se jednoduché vysvětlení -- každý z rezistorů byl zřejmě vytvořen jinou pastou. 

Při testování výkonové zatížitelnosti rezistorů jsme ověřili náš původní předpoklad a také informace získané simulací na počítačových cvičeních. Čím větší je pouzdro součástky, tím lépe odvádí přebytečné teplo, v případě substrátu FR4 dosáhla menší součástka teploty \qty{91}{\degreeCelsius}, což je o \qty{6}{\degreeCelsius} více, než součástka větší. Na keramickém substrátu je teplo obecně odváděno mnohem účinněji a maximální dosažené teploty pro obě součástky byly pouze \qty{44}{\degreeCelsius}. Pro menší součástku dochází ale k většímu ohřevu substrátu, takže při vyšší zástavbové hustotě by i zde hrála velikost pouzdra roli. 

Nakonec jsme destruktivním testem pozorovali změnu SMD rezistoru při přiložení nadměrného napětí. Náš vzorek vydržel (i když už s nevratným poškozením) až do napětí \qty{26,9}{V}. Bylo by zajímavé vyhledat v datasheetu, pro jaký rozsah zatížení byl rezistor určen, a porovnat je se zjištěnými. 