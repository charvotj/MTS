V této úloze jsme měřili nastavený teplotní profil pro reflow pájení. Z naměřených dat je viditelné, že na spodní straně DPS je teplota o něco nižší (při chlazení pak naopak vyšší) z důvodu horšího přenosu tepla. Pro zónu 3 dochází k droobnému překročení nastavené teploty, pokud by bylo překročení ještě vyšší, bylo by potřeba nastavenou teplotu o něco snížit. Gradienty nárustu a poklesu teplot splňují požadované podmínky a teplotní profil je tedy pro pájení vyhovující. 

S nastaveným teplotním profilem jsme provedli testovací zapájení několika SMD rezistorů, následná optická kontrola ukázala, že pájení proběhlo v pořádku. 

Dále jsme se věnovali průzkumu jevu leatching. Pro nastavenou relativně vysokou teplotu \qty{250}{\degreeCelsius} nakonec došlo k leatchingu u všech třech použitých pájecích past. Nejagresivněji působila pájecí slitina Sn42Bi58. Slitina SAC305 prokázala nejepší vlastnoti z hlediska homogenity, jelikož vytvořila pouze jednu kuličku, čímž prošla tzv. solder balling testem. Ovšem z vodivé cesty se přemístila zcela do volného prostoru, což bylo pro nás dosti neočekávané chování. 