Desky plošných spojů (DPS) hrají klíčovou roli ve funkčnosti elektrických zařízení. Vlivem špatně nastaveného výrobního procesu nebo používáním elektronického zařízení může dojít k poškození DPS. V případě méně rozsáhlého poškození je možné tyto vady opravit a DPS nadále využívat. Vždy je potřeba zhodnotit ekonomický aspekt opravy, zeména poměr ceny opravy vůči výrobě nového kusu. 

Procesy výroby oprav různých částí DPS jsou řízeny normami IPC. Kvalitě pájených spojů se věnuje norma IPC-A-610, metodám opravy DPS a požadavkům na kvalitu oprav se věnuje norma IPC 7711/21.


\subsection{Osazování SMD součástek}

Osazování SMD součástek je důležitou součástí výroby elektronických zařízení a sestav. Spávné osazení a zarovnání součástek umožňuje předejít prolémům v další části výrobního procesu. K tomuto účelu slouží zařízení Pick and Place, která umožňují osazování součástek poloautomaticky nebo automaticky.

\subsubsection{Typy osazovacích zařízení}

\begin{itemize}
    \item \textbf{Manuální} - ruční osazování vakuovou
    \item \textbf{Poloautomatická} - osazování probíhá opět ručně, správná poloha součástek je ale signalizována např. za pomoci laseru. Vhodné pro prototypovou a malosériovou výrobu.
    \item \textbf{Automatická} - osazování probíhá automaticky s různými možnostmi sesazení: 
    \begin{itemize}
        \item pevně dané souřadnice (bez sesazení)
        \item mechanické sesazení
        \item laserového sesazení 
        \item vizuálního sesazení s využitím kamerového systému
    \end{itemize}
\end{itemize}

Osazovací metoda závisí na sériovosti výroby a technických specifikacích jednotlivých součástek a DPS.


\subsection{Opravy desek plošných spojů}

Desky plošných spojů představují komplexní síť vodivých cest, pájecích plošek a prokovů, které zajišťují elektrické propojení mezi elektronickými součástkami. Během návrhu, výroby a používání mohou vzniknout různé opravitelné i neopravitelné vady. 

Mezi příčiny opravitelných vad patří:

\begin{itemize}
    \item \textbf{Elektronické součástky} (většinou nutná výměna)
    \item \textbf{Pájené spoje}
    \item \textbf{Pájecí plošky}
    \item \textbf{Kontaktní plošky konektoru}
    \item \textbf{Vodivé cesty}
    \item \textbf{Prokovy}
    \item \textbf{Nepájivá maska}
\end{itemize}

Rozsah poškození a finanční náklady ovlivňují proveditelnost opravy. Je důležité vyhodnotit nejen možnost výměny součástky, ale také časovou a finanční náročnost celého procesu. Dalším důležitým faktorem je unikátnost DPS, která může ovlivnit proveditelnost opravy. 

\subsubsection{Diagnostika vad a kategorie poškození}

Základní důvody poškození mohou být:

\begin{enumerate}
    \item Fyzické poškození - často spojené s manipulací a používáním zařízení. Mohou vzniknout tlakem, pádem nebo demontáží.
    \item Selhání součástky nebo komponenty - způsobené stářím, špatnou výrobou, přehřátím nebo vznik studených spojů.
    \item Poškození vodivé cesty nebo pájecí plošky - viditelné jako fyzické poškození nebo opálení v důsledku přetížení nebo kontaminace.
    \item Špatný návrh - vady v návrhu, které se projevují během výroby nebo používání.
\end{enumerate}

Diagnostika a kategorizace vad je klíčová pro určení postupu opravy, který je pak dále definován normou IPC 7711/21.

