V této laboratorní úloze jsme získali spoustu užitečných znalostí z oblasti oprav desek plošných spojů. Díky praktické povaze úlohy jsme také měli možnost tyto poznatky aplikovat, což může být pro budoucí elektrotechnickou praxi cennou zkušeností. 

Všechny námi provedené opravy byly úspěšné, což jsme ověřili jak optickou kontrolou, tak i kontrolou vodivosti za pomoci multimetru. 

Závěrem jsme byli seznámeni s použitím automatického osazovacího zařízení (ERSA HR600). Toto zařízení je vhodné zejména pro zautomatizování a zkvalitnění opravárenského procesu, neboť při jednom nastavení dokáže osazovat a pájet pouze jednu konkrétní součástku na konkrétní pozici, což je možné využít při sérii stejných oprav. 