Jak je vidět z grafů a tubulky naměřených hodnot, pro některé série jsou výsledky v různých kvadrantech obdobné, pro některé naopak vůbec. Nejhomogennějšíh výsledku dosáhla série RX\_3. Také je vidět, že napříč sériemi vycházely celkově o něco menší hodnoty v prvním kvadrantu a naopak vyšší ve druhém a třetím. 

Při porovnání naměřených hodnot s teoretickými se ale dostáváme do prekérní situace. Odchylka zde dosahuje několika řádů kdy namísto stovek \unit{\ohm} měříme hodnoty i v nižších stovkách \unit{\mega\ohm}, typycky pak v jednotkách \unit{\mega\ohm}. Toto nasvědčuje buďto hrubé systematické chybě měření neno špatným informacím ohledně použité odporové pasty, popř. kombinaci obou faktorů. Vzhledem k tomu, že měření bylo prováděno poměrně dlouhý čas a dohlíženo čtyřmi osobami se mi takto hruhá chyba jeví jako nepravděpodobná, ovšem vyloučna není. 