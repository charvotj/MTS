Teorie potřebná k této úloze vychází z předešlých laboratorních úloh, zejména úlohy 2, kdy jsme měřili hodnoty tlustovrstvých rezistorů v závislosti na různých podobách výpalu a také změnu jejich odporu v závislosti na teplotě. Přijde mi zbytečné uvádět znovu základní informace o samotné technologii nebo výpočtu odporu na čtverec, které již v této době semestru musí znát opravdu každý student, proto bude tato sekce o něco kratší.

\subsection{Dostavování TLV rezistorů}
Co se týče nové teorie věnuje se tato úloha také dostavování (nebo také trimmování) již natištěných rezistorů. 
Jak jsme si již vyzkoušeli na předchozích úlohách, přesnost tisku je velkou neznámou a i při optimalizaci všech dostupných parametrů jsou hodnoty stále poměrně nepřesné. Z tohoto důvodu se s nepřesností v návrhu počítá a tisknou se rezistory s o něco nižšími hodnotami než je požadováno. Následně jsou rezistory měřeny a v průběhu měření je jich část odebrána tak, aby se svou hodnotou více přiblížili požadavku. 

Nejčastějšími způsoby je mechanické osbroušení, obvykle proudem částic korundu nebo křemíku, nebo odpaření vrstvy laserem, většinou typ YAG nebo CO\(_{2}\). 

Pro dostavení je možné vyřezat do tlustovrstvého motivu různé obrazce. Obvykle se volí přímý řez od kraje směrem ke středu. Pokud se takovýchto provede více vedle a naproti sobě, vznikne serpentýnový motiv. Alternativou je ještě výřez ve tvaru L \cite{zadani,schroeder2022trimming}.