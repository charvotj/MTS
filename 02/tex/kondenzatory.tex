V této části jsme měřili hodnoty kapacit kondenzátorů vytvořených za pomoci tlustovrstvé technologie, konkrétně s použitím dielekrické pasty ESL 4917. Měřeny byly opět LCR metrem (BK Precision 880)
Abychom ověřili přesnost těchto kondenzátorů, je potřeba vypočítat teoretickou hodnotu kapacity, které by měl ideální kondenzátor dosáhnout. Vyjdeme ze známého vztahu:
\[
    C=\varepsilon_{0}\cdot\varepsilon_{r}\cdot\frac{S}{d}
\] 




Katalogový list použité pasty \cite{pastaDatasheet} uvádí pro permitivitu a typickou tloušťku vrstvy vždy rozsah hodnot, viz Tab.~\ref{tab:pasta_hodnoty_katalog}. Pro teoretický výpočet kapacity použijeme tedy vždy okrajové hodnoty:
\[
    C_{teor MAX} = \varepsilon_{0}\cdot\varepsilon_{r MAX}\cdot\frac{S}{d_{MIN} }
\]
\[
    C_{teor MIN} = \varepsilon_{0}\cdot\varepsilon_{r MIN}\cdot\frac{S}{d_{MAX} }
\]

\begin{table}[h!]
    \caption{Výňatek z katalogového listu pasty ESL 4917.}
    \centering
    \def\arraystretch{1.4}
    \begin{tabular}{l|l}
        \(d_{MIN} \)            & \qty{35}{\micro\meter} \\ \hline
        \(d_{MAX} \)            & \qty{50}{\micro\meter} \\ \hline
        \(\varepsilon_{MIN} \)  & 8                      \\ \hline
        \(\varepsilon_{MAX} \)  & 11                     \\ 
        \end{tabular}
    \label{tab:pasta_hodnoty_katalog}
\end{table}

Dále jsme pro každý kondenzátor vypočetli kapacitu na čtverec. Stejný postup byl opakován ve dvou variantách, nejprve pro kondenzátory s vrstvami vypálenými současně (viz Tab. \ref{tab:c_najednou_hodnoty}), poté pro kondenzátory s vrstvami vypalovanými postupně (viz Tab. \ref{tab:c_postupne_hodnoty}). 

Naměřené a vypočtené hodnoty pro obě série měření se nacházejí v Tab.~\ref{tab:c_najednou_hodnoty} a \ref{tab:c_postupne_hodnoty}.

% C najednou
\begin{table}[h!]
    \caption{Teoretické a měřené hodnoty pro kondenzátory s výpalem najednou.}
    \centering
    \def\arraystretch{1.4}
    \begin{tabular}{l|l|l||l|l}
        S [sq] & \(C_{meas} \) [\unit{\pico\farad}] & \(C_{sq} \) [\unit{\pico\farad\per sq}]  & \(C_{teor MIN}\) [\unit{\pico\farad}] & \(C_{teor MAX}\) [\unit{\pico\farad}]\\ \hline
        4      & 11,7   & 2,93  & 5,667     & 11,13    \\ \hline
        25     & 53,4   & 2,14  & 35,42     & 69,57    \\ \hline
        50     & 104,9  & 2,10  & 70,83     & 139,14   \\ \hline
        100    & 203,0  & 2,03  & 141,67    & 278,27   \\ \hline
        200    & 406,7  & 2,03  & 283,33    & 556,55   \\ 
        \end{tabular}
    \label{tab:c_najednou_hodnoty}
\end{table}
\vspace{100pt}
% C postupně
\begin{table}[t!]
    \caption{Teoretické a měřené hodnoty pro kondenzátory s výpalem postupně.}
    \centering
    \def\arraystretch{1.4}
    \begin{tabular}{l|l|l||l|l}
        S [sq] & \(C_{meas} \) [\unit{\pico\farad}] & \(C_{sq} \) [\unit{\pico\farad\per sq}]  & \(C_{teor MIN}\) [\unit{\pico\farad}] & \(C_{teor MAX}\) [\unit{\pico\farad}]\\ \hline
        4       & 14,3  & 3,58  & 5,67       & 11,13   \\ \hline
        25      & 67,5  & 2,70  & 35,42      & 69,57   \\ \hline
        50      & 133,1 & 2,66  & 70,83      & 139,14  \\ \hline
        100     & --    & --    & 141,67     & 278,27  \\ \hline
        200     & 499,3 & 2,50  & 283,33     & 556,55  \\ 

        \end{tabular}
    \label{tab:c_postupne_hodnoty}
\end{table}