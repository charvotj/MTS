Z měření hodnot odporu můžeme říci, že při postupném výpalu dosahujeme lepších výsledků než při výpalu najednou. Variační koeficient a stejně tak i odchylka od předpokládané teoretické hodnoty vychází vždy menší pro sérii rezistorů s postupným výpalem vrstev. 

Použití krycí vrstvy má také vliv na hodnotu výsledného odporu, při výpalu najednou jej výrazně zvyšuje, zřejme z důvodu difuze krycí izolační vrstvy do nevypálené odporové vrstvy. Při postupném výpalu k difuzi a tedy i zvýšení odporu stále dochází, ale tento efekt je výrazně potlačen.

Při měření rezistorů s postupně se zvětšující plochou (série R5 a R6) jsme také vypozorovali výrazně nižší hodnoty odporu na čtverec pro malé rezistory (nejvýrazněji pro 1 a 2 čtverce). Toto bude dáno nejspíše difuzí vodivých kontaktních plošek do odporové vrstvy. Tento jev je u malých rezistorů srovnatelný s jejich velikostí a projeví se zde tedy poměrně více než u rezistorů větších. Jestli by se tomuto jevu dalo také zabránit postupným výpalem vrstev nám nebylo umožněno zjistit. 


Na základě katalogového listu jsme vypočetli teoretický rozsah kapacity, kterou bychom měli u vytvořených kondenzátorů předpokládat. Většina vytvořených kondenzátorů se opravdu svou kapacitou vejde do tohoto rozsahu, pouze ty nejmenší (4 čtverce, v obou variantách výpalu) hodnoty překračují. Je možné, že zde hraje roli kapacita přívodních vodičů a nebo opět difuze vrstev. Dále by se dala ověřit, jak moc je poměrná kapacita (na čtverec) konzistentní u různých vzorků, pro statistický závěr je ale vzorků příliš málo. 

Při měření grindometrem jsme přibližně stanovili velikosti zrn ve třech vrorcích past. První dva odpovídají tímto parametrem tlustovrstvým pastám, třetí vzorek pak spíše pastě pájecí.