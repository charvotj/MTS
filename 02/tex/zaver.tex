Z měření hodnot odporu můžeme říci, že při postupném výpalu dosahujeme lepších výsledků než při výpalu najednou. Variační koeficient a stejně tak i odchylka od předpokládané teoretické hodnoty vychází vždy menší pro sérii rezistorů s postupným výpalem vrstev. 

Použití krycí vrstvy má také vliv na hodnotu výsledného odporu, při výpalu najednou jej výrazně zvyšuje, zřejme z důvodu difuze krycí izolační vrstvy do nevypálené odporové vrstvy. Při postupném výpalu k difuzi a tedy i zvýšení odporu stále dochází, ale tento efekt je výrazně potlačen.

Při měření rezistorů s postupně se zvětšující plochou (série R5 a R6) jsme také vypozorovali výrazně nižší hodnoty odporu na čtverec pro malé rezistory (nejvýrazněji pro 1 a 2 čtverce). Toto bude dáno nejspíše difuzí vodivých kontaktních plošek do odporové vrstvy. Tento jev je u malých rezistorů srovnatelný s jejich velikostí a projeví se zde tedy poměrně více než u rezistorů větších. Jestli by se tomuto jevu dalo také zabránit postupným výpalem vrstev nám nebylo umožněno zjistit. 


Při testování kondenzátorů jsme nejprve na základě katalogového listu vypočetli teoretický rozsah kapacity, kterou bychom měli u vytvořených kondenzátorů předpokládat. Většina vytvořených kondenzátorů se opravdu svou kapacitou vejde do tohoto rozsahu, pouze ty nejmenší (4 čtverce, v obou variantách výpalu) hodnoty překračují. Je možné, že zde hraje roli kapacita přívodních vodičů a nebo opět difuze vrstev. Dále by se dalo ověřit, jak moc je poměrná kapacita (na čtverec) konzistentní u více různých kondenzátorů, pro statistický závěr je ale vzorků příliš málo. 


Při měření grindometrem jsme přibližně stanovili velikosti zrn ve třech vrorcích past. První dva odpovídají tímto parametrem tlustovrstvým pastám, třetí vzorek pak spíše pastě pájecí.


Při testu vazby na substrát jsme provedli vrypový test na třech vzorcích. Jako nejtvrdší se (dle snímků z mikroskopu) ukázala pasta rezinátová, opticky se rýha zdá nejméně hluboká, v jednom místě také došlo k delaminaci a odštípnutí kousknu vrstvy. Nejměkčí se naopak ukázala pasta polymerová, na snímcích je patrný hluboký vryp a s ním dlouhá stopa odloupnutého materiálu za hranicí plošky. Bohužel jsme si nepoznamenali měřítko mikroskupu, tedy přesnější porovnání cermetové a polymerové pasty není v tuto chvíli možné. 

Co se týče dat ze Scratch Testeru, zjistili jsme, že vyučujícím dodaná data jsou pouze pro některá měření (4 skupinky z celkových 8) a to zřejmě ne pro to naše. Také se podle průběhu křivek (přítomnost nebo naopak nepřítomnost "hrbů", které zřejmě značí přechod hrotu přes část naneseného motivu -- plošku) zdá, že byla prohozena data pro rezinátovou a některou z dalších past. Z takto nejasných dat, které navíc ani nemohou korespondovat s fotografiemi z mikroskopu, by bylo pošetilé vznášet jakékoliv závěry.