\subsection{Čtyřbodová metoda}
Jedná se o relativně přesnou metodu měření vrstvového odporu. Je možné ji použít jak při kontrole kvality polovodičů, tak i při testování TLV odporů \cite{zadani}. Metoda je použitelná pouze pro vzorky s podstatně většími rozměry, než je vzdálenost měřících hrotů \cite{hrabovsky2012}. 

Jako měřící prostředek slouží čtveřice hrotů umístěných nejlépe v jedné linii. Mezi krajními hroty necháme protékat definovaný proud a mezi prostředními hroty pak měříme úbytek napětí, který tento proud vyvolá. 

Pokud neměříme vzorky s nekonečnou plochou, je potřeba měření kompenzovat různými korekčními faktory. Zejména kompenzujeme geometrické rozměry a tvar vzorku, k tomu je možné využít např. van der Pauwovu metodu \cite{zadani,hrabovsky2012}. Pro časté tvary a rozměry vzorků pak můžeme vytvořit tatulku kompenzačních koeficientů. 